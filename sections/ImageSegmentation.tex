\section{Image Segmentation\buch{Ch. 10}\buchSeite{689-794}}
\paragraph{Preliminaries}
\begin{itemize}
\item Separate the image into different parts, i.e. regions or objects
\item Often a scene can be designed that the segmentation task is simplified
\item Segmentation is based on some property of the scene
\begin{itemize}
\item Intensity discontinuity
\item Intensity similarity
\item More complex properties: Motion, Texture, Shape, Color, etc.
\end{itemize}
\end{itemize}

\subsection{Fundamentals\buchSeite{690}}
\begin{multicols}{2}
\begin{enumerate}[label={\alph*)}]
\item $\bigcup\limits_{i=1}^{n}R_i=R$
\item $R_i$ is a connected set, $i=1,2,\dots ,n$
\item $R_i\bigcap R_j = 0$ for all $i$ and $j, i\neq j$
\item $Q(R_i)=$ TRUE for $i=1,2,\dots ,n$
\item $Q(R_i\bigcup R_j)=$ FALSE for any adjacent regions $R_i$ and $R_j$
\end{enumerate}
\vfill
\columnbreak
\begin{itemize}
\item The entire spatial region that is occupied by an image is called $R$
\item Image segmentation is the task of partitioning $R$ into sub-regions $R_i$
\item $Q()$ is a measure of similarity of a particular property
\item d) must be true, since it defines the region
\item e) must be false, since otherwise, the regions should have been merged
\end{itemize}
\end{multicols}

Focus on the intensity values
\begin{itemize}
\item If there are discontinuities, a new region starts.
\item If the intensity does not vary much, then this is one region.
\end{itemize}

\subsection{Point, line and edge detection\buchSeite{692}}
When the intensity changes abruptly in a local neighborhood, this indicates an edge. It can be a simple point or a line.

\begin{multicols}{2}
\subsubsection{Background\buchSeite{692}}
Abrupt local changes can be detected using derivatives.\\
First order derivative
\[
	\frac{\partial f}{\partial x} = f'(x)=f(x+1)-f(x)
\]
The second order derivative
\[
	\frac{\partial^2f}{\partial x^2}=\frac{\partial f'(x)}{\partial x}=f''(x)=f(x+1)+f(x-1)-2f(x)
\]
This can be calculated using a spatial filter.
\[
	\begin{bmatrix}
	w_1, w_2, w_3\\
	w_4, w_5, w_6\\
	w_7, w_8, w_9
	\end{bmatrix}
\]

\columnbreak

\begin{itemize}
\item First order in y direction \\ $w_6=1, w_5=-1$
\item First order in x direction \\ $w_8=1, w_5=-1$
\item Second order in y direction \\ $w_6=1, w_4=1, w_5=-2$
\item Second order in x direction \\ $w_8=1, w_2=1, w_5=-2$
\end{itemize}
\ \\
\end{multicols}
\subsubsection{Detection of Isolated Points\buchSeite{696}}
The Laplacian combines the second derivatives in both spatial directions. This results in
\[
	\nabla^2f(x,y)=f(x+1,y)+f(x-1,y)+f(x,y+1)+f(x,y-1)-4f(x,y)
\]
It can be extended to also include the diagonal terms wich results in the following mask
\[
	\begin{bmatrix}
	 1 & 1 & 1\\
	 1 & -8 & 1\\
	 1 & 1 & 1
	\end{bmatrix}
\]
The Laplacian is isotropic with respect to $0^\circ$, $45^\circ$, $90^\circ$ and $135^\circ$. \\

Point detection is now achieved by filtering the image with the mask and thresholding this result image $R$.
\[
	g(x,y) = 
	\begin{cases}
		1 & \text{if } |R(x,y)| \geq T \\
		0 & \text{otherwise}
	\end{cases}
\]

\subsubsection{Line Detection\buchSeite{697}}
Often, a line of a known direction should be detected. We use special masks for that particular direction
\begin{figure}[H]
	\centering
	\begin{subfigure}[b]{0.2\textwidth}
		\centering
		\begin{tikzpicture}
			\matrix[3x3mask]{
				-1 & -1 & -1 \\%
		    	2  & 2  & 2 \\%
				-1 & -1 & -1\\};
		\end{tikzpicture}
		\caption{Horizontal}
	\end{subfigure}
	\begin{subfigure}[b]{0.2\textwidth}
		\centering
		\begin{tikzpicture}
			\matrix[3x3mask]{
				2  & -1 & -1 \\%
				-1 & 2  & -1 \\%
				-1 & -1 & 2 \\%
			};
		\end{tikzpicture}
		\caption{$+45^\circ$}
	\end{subfigure}
	\begin{subfigure}[b]{0.2\textwidth}
		\centering
		\begin{tikzpicture}
			\matrix[3x3mask]{
				-1  & 2 & -1 \\%
				-1 & 2  & -1 \\%
				-1 & 2 & -1 \\%
				};
		\end{tikzpicture}
		\caption{Vertical}
	\end{subfigure}
	\begin{subfigure}[b]{0.2\textwidth}
		\centering
		\begin{tikzpicture}
			\matrix[3x3mask]{
				-1 & -1 & 2 \\%
				-1 & 2  & -1 \\%
				2  & -1 & -1 \\%
			};
		\end{tikzpicture}
		\caption{$-45^\circ$}
	\end{subfigure}
	\caption{Line detection masks}
\end{figure}

\subsubsection{Edge Models}
The typical intensity profile of an edge is shown in figure \ref{fig:imseg_edgemodels}.
Note that the derivatives are very sensitive to noise. Usually the image is smoothed (low pass filtered) before
edge detection.

\newpage
\subsubsection{Edge localization\buchSeite{706}}
\begin{multicols}{2}
The previously shown method generates edge points. The goal is to keep the ones which truly belong to an edge.
Since first order derivatives are helpful, a nice way of combining the two partial derivatives into one value is the magnitude of the gradient.
\columnbreak
\begin{equation}
\nabla f = \text{grad}(f) = 
\begin{bmatrix} g_x \\ g_y \end{bmatrix} 
= \begin{bmatrix} \frac{\partial f}{\partial x} \\[4pt] \frac{\partial f}{\partial y} \end{bmatrix}\notag
\end{equation}
\end{multicols}
The gradient (which is a vector) points into the direction of the greatest rate of change of $f$ at the location $(x,y)$ and is therefore orthogonal to the direction of the edge, which has no change.
The magnitude of the gradient vector is the value of the rate of change in that direction.
\begin{figure}[H]
	\centering
	\begin{subfigure}[b]{0.2\textwidth}
		\centering
		\begin{tikzpicture}
			\matrix[3x3mask]{
				-1 & -1 & -1 \\%
				0  & 0 & 0 \\%
				1  & 1 & 1 \\%
			};
		\end{tikzpicture}
		\caption{Prewitt}
	\end{subfigure}
	\begin{subfigure}[b]{0.2\textwidth}
		\centering
		\begin{tikzpicture}
			\matrix[3x3mask]{
				-1 & 0 & 1 \\%
				-1 & 0 & 1 \\%
				-1 & 0 & 1 \\%
			};
		\end{tikzpicture}
		\caption{Prewitt}
	\end{subfigure}
	\begin{subfigure}[b]{0.2\textwidth}
		\centering
		\begin{tikzpicture}
			\matrix[3x3mask]{
				-1 & -2 & -1 \\%
				0 & 0 & 0 \\%
				1 & 2 & 1 \\%
			};
		\end{tikzpicture}
		\caption{Sobel}
	\end{subfigure}
	\begin{subfigure}[b]{0.2\textwidth}
		\centering
		\begin{tikzpicture}
			\matrix[3x3mask]{
				-1 & 0 & 1 \\%
				-2 & 0 & 2 \\%
				-1 & 0 & 1 \\%
			};
		\end{tikzpicture}
		\caption{Sobel}
	\end{subfigure}
	\caption{Gradient operators: Prewitt and Sobel masks}
\end{figure}

\newpage